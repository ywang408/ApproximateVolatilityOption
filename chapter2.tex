\chapter{Method Description}\label{ch2}

In this section, we will discuss HP's DOI method and KM's approximation method. In section \ref{sec: 2.1}, we introduce HP's DOI method on the approximation of solutions of parabolic PDEs; In section \ref{sec: 2.2}, we illustrate complementary contents for KM's approximation method.

\section{The DOI Variance Reduction Method}
\label{sec: 2.1}

Consider a multi-factor model, in which a $d$-dimensional vector of state variables $X_t$ on a filtered probability space$(\Omega,\mathcal F, \mathbb Q)$ satisfies the following stochastic differential equations(SDEs)

\begin{equation}\label{general model}
    dX_t= \mu(t, X_t) dt + \sigma(t, X_t) dW_t
\end{equation}

\noindent where $\mu(t,X_t)$ and $\sigma(t, X_t)$ are drift and diffusion functions under the risk-neutral measure $\mathbb Q$. HP imposes $\mu: [0,T] \times \Gamma \rightarrow \mathbb R^d$, $\sigma: [0,T] \times \Gamma \rightarrow \mathbb R^d$, in which $\Gamma$ is assumed to be a bounded subset of $\mathbb R^d$. $\mu$ and $\sigma$ also satisfies appropriate growth and Lipschitz conditions such that equation(\ref{general model}) admits a unique strong solution and is Markovian; $W_t$ is a $d$-dimensional standard Brownian Motion and $t \in [0,T]$.

Let $w(t,x)$ be the valuation function satisfying $w: [0,T] \times \Gamma$, payoff function $h(t,x)$ satisfying $h: B \rightarrow \mathbb R, B \subseteq \mathbb R^d$. Define the infinitesimal generator $L$ associated with equation \eqref{general model} to be

\begin{equation}\label{general inf gen}
    \begin{aligned}
        L w(t, x)&=\frac{\partial w}{\partial t} + \sum_{i=1}^{d} \mu_i(t, x) \frac{\partial w}{\partial x}+\frac{1}{2} \sum_{i=1}^{d}\sum_{j=1}^{d} (\sigma(t,x) \sigma^{\intercal}(t,x))_{i,j} \frac{\partial^2 w}{\partial x_i x_k}
    \end{aligned}
\end{equation}

Let $R(t,x)$ be the instantaneous short-term interest rate also satisfying $R: B \rightarrow \mathbb R$, combining with equation \eqref{general model} and equation \eqref{general inf gen}, the task is to find an approximation to the solution to the following partial differential equation (PDE)

\begin{equation}\label{pde under general}
    Lw(x,t) = R(x,t)w(x,t)
\end{equation}

\noindent with boundary condition $w(t,x) = h(t,x)$. It's easily seen that under risk neutral measure $\mathbb Q$, the instantaneous option price change is equal to the price gain in saving account. 

Next we consider to use a $m$-dimensional($m \leq d$) process $\bar{X}(t)$ which is a simpler auxiliary model to approximate the price of option. $\bar{X}(t)$ satisfies the following SDE

\begin{equation}\label{approx model}
    \begin{aligned}
        &d\bar{X}_t= \begin{cases}   \bar{\mu}_i(t, \bar{X}_t) dt + \bar{\sigma}_i(t, \bar{X}_t) dW_t & 1 \leq i \leq m \\
        0 \cdot dt + 0 \cdot dW_t & m < i \leq d \end{cases}
        \end{aligned}
\end{equation}

\noindent where $\bar{\mu}(t, \bar{X}_t)$ and $\bar{\sigma}(t, \bar{X}_t)$ are drift and diffusion functions, and they are also assumed to satisfy appropriate conditions such that equation(\ref{approx model}) admits a unique strong solution and is Markovian. In KM's method, it is automatically satisfied because they suppose that there always exists a closed-from solution to auxiliary model such that Ito-Taylor expansions can be applied further.

Let $\bar{w}(t,x)$ be the option price written on process $\bar{X}_t$, the infinitesimal generator $\bar{L}$ for option price $\bar{w}$ is the same as equation(\ref{general inf gen}) but replacing $\mu(t,x)$, $\sigma(t,x)$ by $\bar{\mu}(t,x)$ and $\bar{\sigma}(t,x)$. Therefore $\bar{w}(t,x)$ is a solution to

\begin{equation}\label{pde under approx}
    \mathcal{\bar{L}}\bar{w}(x,t) = R(x,t)\bar{w}(x,t)
\end{equation}

Denote the price difference $\Delta w(t,x) = w(t,x) - \bar{w}(t,x)$, then $\Delta w$ satisfies 

\begin{equation}
    L \Delta w(t, x)+ (L-\bar{L}) w(t,x)=R(x, t) \Delta w(t, x)
\end{equation}

\noindent with boundary condition $\Delta w(t,x) = 0$. Next we define

\begin{equation}\label{mis-pricing}
    \begin{aligned}
        \delta(t,x) &= (L-\bar{L}) \bar{w}(t,x) \\
        &=\sum_{i=1}^{d} \Delta \mu_{i}(t,x) \frac{\partial \bar{w}(t,x)}{\partial x_{i}}+\frac{1}{2} \sum_{i, j=1}^{d} \Delta \sigma_{i j}^{2}(x, t) \frac{\partial^{2} \bar{w}(x,t)}{\partial x_{i} \partial x_{j}}
    \end{aligned}
\end{equation}

\noindent using the Feynman-Kac representation leads to option price $w(t,x)$ taking the following form

\begin{equation}\label{feynman-kac rep}
        w(t, x)=\bar{w}(t,x)+\int_{t}^{T} \mathbb{E}^{t,x}\left[\exp \left(-\int_{t}^{s} R(u, X_u) d u\right) \delta(s,X_s)\right] d s
\end{equation}

\noindent Finally, under the initial condition $Z_0 = w(0,x)$, the DOI estimator

\begin{equation}
        Z_t =\bar{w}(t, x)+\int_{t}^{T} \exp \left(-\int_{t}^{s} R(u, X_u) d u\right) \delta(s,X_s)d s
\end{equation}

\noindent is an unbiased estimator for $Z_0$. And if a good auxiliary model is chosen, for example, when $\bar{L}$ is close to $L$, the variance of $Z_t$ will be small.

\section{KM's approximation Method based on the DOI method}
\label{sec: 2.2}

Recall equation\eqref{feynman-kac rep}, this equation is the Theorem 1(Asset Price Representation)\footnote{Except from equation \eqref{mis-pricing}, KM also consider another mispricing function $d(t,x)=h(t,x)-\bar{h}(t,x)$ arising from using a wrong payoff function $\bar{h}(t,x)$, which is useful to apply his method on bond pricing. In this paper we mainly discuss option pricing thus we relax his condition here.} in \cite{kristensen_adding_2011}, KM names equation \eqref{mis-pricing} the mispricing function. His method is to look for an approximation of the error term in order to adjust the price $\bar{w}(t,x)$ for the error due to the use of the auxiliary model to price options.

We first introduce Ito-Taylor expansion, for sufficiently smooth function $f(t,x)$, it is given by

\begin{equation}
    \mathbb{E}^{t, x}[f(s, X_s)]=\sum_{N=0}^{J} \frac{(s-t)^{N}}{N !}L^N f(t, x)+\mathcal{R}_{J}
\end{equation}

\noindent where $L$ is the infinitesimal generator we have defined in equation \eqref{general inf gen}, the remainder term $\mathcal{R}_{J}$ is given by

\begin{equation}
    \mathcal{R}_{J}=\mathbb{E}^{t, x}\left[\int_{t}^{s} d u_{1} \int_{t}^{u_{1}} d u_{2} \cdots \int_{t}^{u_{J}}L^{J+1} f\left(u_{J+1}, X\left(u_{J+1}\right)\right) d u_{J+1}\right]
\end{equation}

For Ito-Taylor expansion to hold, KM imposes stronger assumptions for infinitesimal generator $L$:

\begin{enumerate}
    \item First, $L$ has a transition density $p_t(y|x)$ with respect to Lebesgue measure.
    \item $L$ has an invariant measure $\pi$ satisfying: $\pi(x)p_t(y|x) = \pi(y)p_t(x|y)$.
\end{enumerate}

\noindent The first condition requires the considered diffusion model to have a transition density, this is satisfied in most financial models; The second condition is a generalization of time-reversibility. In particular, if the process is univariate and stationary, it is necessarily time-reversible and satisfies the second condition.

Assume closed form solution of auxiliary model $\bar{w}$ under auxiliary model is sufficiently smooth. In other words, for $N \geq 1$, assume $\delta(t,x)$ to be $2N$ times differentiable with respect to $x$, and to be $N$ times differentiable with respect to $t$. By applying Ito-Taylor expansion to equation \eqref{feynman-kac rep}

\begin{equation}
    w(t, x)=\bar{w}(t,x)+\int_{t}^{T} \mathbb{E}^{t,x}\left[\exp \left(-\int_{t}^{s} R(u, X_u) d u\right) \delta(s,X_s)\right] d s
\end{equation}

\noindent Then KM proposes Definition 1(Asset Price Approximation) in his paper, it's given by

\begin{equation}\label{approx formula}
    w_{N}(t, x)=\bar{w}(t,x)+\sum_{n=0}^{N} \frac{(T-t)^{n+1}}{(n+1) !} \delta_{n}(t, x)
\end{equation}

\noindent where $\delta_0$ is the original mispricing function in equation \eqref{mis-pricing}, and

\begin{equation}
    \begin{aligned}
        &\delta_0(t,x) = \delta(t,x) \\
        &\delta_{n}(t, x)=L \delta_{n-1}(t, x)-R(t, x) \delta_{n-1}(t, x)
        \end{aligned}
\end{equation}

KM refers in his paper that the Ito-Taylor in equation \eqref{approx formula} can be calculated once for all, meaning once implemented, it requires no computation time. However, in practice, the number of terms in equation \eqref{approx formula} grows exponentially as we keeping expansion. When the expansion order $N$ is large, we must rely on symbolic languages to solve the approximation formula, the advantage of this method is constrained because process of solve for derivatives and further calculation can be very long. Example is discussed in our chapter \ref{ch4}.

At last, different from the DOI method, KM's approximation method, is based on the combination of price of auxiliary model and the expansion of mispricing functions, which leads to a nuisance parameter. This parameter does not affect the unknown price, but does enter the pricing formula for the auxiliary model. For example, if we use Black-Scholes model as auxiliary model to price options under Heston model, 

$$
\begin{aligned}
    d S_{t}&=\mu S_{t} d t+\sqrt{v_{t}} S_{t} d W_{t}^{1} \\
    d v_{t}&=\kappa\left(\theta-v_{t}\right) d t+\xi \sqrt{v_{t}} d W_{t}^{2}
\end{aligned}
$$

\noindent The mispricing function here is $\delta = \frac{1}{2}(v(t)-\sigma_0)S^2 \frac{\partial^2 w^{\text{bs}}}{\partial S^2}. $ Then the nuisance parameter is the instantaneous volatility $\sigma_0$ in Black-Shcoles model. KM proposes that when we approximate $w(x,t)$ with $w_N(x,t;\hat{\sigma_0})$, we consider

$$
\hat{\sigma}_{N}(x, t)=\arg \min _{\sigma}\left(w_{N}(x, t ; \sigma)-w_{0}(x, t ; \sigma)\right)^{2}
$$

\noindent where $w_{0}(x, t ; \sigma)\equiv w^{\text{bs}}(x, t ; \sigma)$. Thus we can set instantaneous volatility to be the spot value of volatility in Heston model, or be the long-time mean of Heston model, that's is, $\sigma_0 = v(t)$ or $\sigma_0 = \theta$. In our case, the setting of nuisance parameter is specified in chapter \ref{ch4}.