\chapter{Introduction}

The Volatility, as a measurement of the risk, is one of the most important element in the study of finance. \cite{whaley_derivatives_1993} proposes that by allowing investors to hedge directly against shifts in volatility, these securities enable investors to avoid the costs of dynamically adjusting positions for changes in volatility and serve to make the market more complete. During the past few years, derivatives written on volatility have developed rapidly, in which options are clearly the fundamental type of derivatives. However, unlike options on stocks, pricing volatility options can be challenging by its mean-revering property. On the basis of \cite{cox_theory_1985}, \cite{hull_pricing_1987}, \cite{heston_closed-form_1993}, \cite{grunbichler_valuing_1996} proposes Mean-Reverting Square Root(MRSR) model; \cite{chan_empirical_1992} then generalizes Mean-Reverting constant elasticity of variance(MRCEV) model and \cite{gatheral_consistent_2008} proposes the Double CEV model to capture the dynamics of volatility. \cite{grunbichler_valuing_1996} gives a closed-form solution for options under MRSR model, but there're no existing solutions to the following two models. The aim of our paper is to use the solution of square-root mean-reverting model to approximate the other two \footnote{Strictly speaking, for double CEV model, we mainly consider Heston plus CEV model, that is $\gamma_1=\frac{1}{2}$. Details are discussed in \ref{sec: 3.2}}{models}.

\cite{heath_variance_2002}(HP) first introduce a diffusion operator integral (DOI) method, he uses an auxiliary model and then apply Ito calculus on it to find unbiased variance reduction estimators for the true model, he also adapt DOI method to be employed in conjunction with PDE methods and test his method with Heston model. Similar to the idea of HP, \cite{kristensen_adding_2011}(KM) doesn't use auxiliary model to serve as a variance reduction estimator, but apply Ito-Taylor expansions on the difference of auxiliary model and true model to  create increasingly improved refinements. KM apply his method on several fields including bond pricing under CIR model and option pricing under stochastic volatility models.

Based on HP and KM's work, we extend their methods to price options under mean-revering models. This paper is organized as following, in chapter \ref{ch2}, we illustrate DOI method and KM's expansion method; In chapter \ref{ch3}, we discuss our method on approximating option prices under mean-reverting CEV model and Heston plus CEV model, including the auxiliary model selection, techniques of taking partial derivatives; In chapter \ref{ch4}, we show our numerical results and we put our conclusion in chapter \ref{ch5}.