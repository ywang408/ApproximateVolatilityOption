%% stevensthesis_num.tex
%% Copyright 2008 B. E. Arnett
%
% ----->  Latex is not required for writing your dissertation or thesis.  Most theses in liberal arts and other areas are done in Word or WYSIWYG editors
%------>  Latex is frequently used in math, physics and computer science for ease in producing formulas.  
%------>  This template is provided for Latex users to help in the formatting of the thesis or dissertation.  
%------>  This work is not maintained by Stevens Library anymore and is provided on an: "as is" basis.
%
% This work may be distributed and/or modified under the conditions of the LaTeX Project Public License, either version 1.3 of this license or (at your option) any later version.
% The latest version of this license is in http://www.latex-project.org/lppl.txt and version 1.3 or later is part of all distributions of LaTeX version 2005/12/01 or later.
%
% This work has the LPPL maintenance status `maintained'.
% 
%
% This work consists of the files stevensthesis_num.tex and the derived file pgnumchapter_nums.sty.

%
% This file may serve as a template for dissertation / thesis submission for Stevens Institute of Technology.  
%
% There are fields in titlepg.sty that must be updated with your own information.  The file pgnumchapter.sty must be included but  does not need to be updated, unless you want to change the layout of your pages and chapters.  
%
% Both dissertations and master thesis can be created from this program.  
% These fields should be changed:
%    {thesistype} should be either "dissertation" or "thesis" 
%    {thesisdegree} should be either "doctor of philosophy" for dissertation or
%            "Master of Engineering - Electrical Engineering" 
%            "Master of Science - Computer Science"   or the correct degree designation.
%
%
% Please see the formatting instructions on the library website at
%      https://library.stevens.edu/services/submit-theses-dissertations/dissertation-submission
%
% This template was created by Barbara Arnett, when she was part of the library staff
%
% update January 2009 - updated font size to 12p and added subsections to show up in the table of contents 
% update April 2009 - added optional dedication page, code courtesy of James Weatherall
% update May 2009 - added code to include the abstract, acknowledgment, dedication and vita to the table of contents.  As per Doris Oliver, everything except the title page copyright page, and table of contents should appear in the table of contents.
%
% update July 2009 - added code to pgnumchapter_nums.sty to have top of table of contents, figures & symbols start 1.5 inches from the top.
%
% update February 2010 - added \bibliography and BibTex file 
%
% update May 2010 - added line to ensure Table of Contents prints in normalsize font when run in Linux
%                            also changed copyright text to say C YYYY, name. All rights reserved.
% update February 2011 - removed line \renewcommand{\thesubsection}{\thesection\arabic{subsection}}
%            to allow toc subsections to print normally (see bea02152011)
% update 2014 Barbara Arnett, web services librarian, stevens institute of technology
%           is no longer maintaining the template - SIT library is not supporting the template anymore
%
% update March 2018 by Honglei Zhao at the time Ph.D. candidate in Financial Engineering 
%         - adding some packages commonly used in theses formatting. Comments are added to the right of the 
%           package call explaining the purpose 

\documentclass[12pt]{report}
\usepackage {fancyhdr}
\usepackage{graphicx, amssymb, changepage}
\usepackage{rotating}
\usepackage{setspace}
\usepackage{pgnumchapter_nums}
\usepackage{titlepg}
\usepackage{multirow} % multi row in tables
\usepackage{float}% for table and figure enforce placement, the [H] option
\usepackage{titlesec}% more layers of sections
\usepackage{bbm} %bold in math
\usepackage{subfig}% subfigs
\usepackage{array}%include tables with first row bold
\newcolumntype{@}{>{\global\let\currentrowstyle\relax}}
\newcolumntype{^}{>{\currentrowstyle}}
\newcommand{\rowstyle}[1]{\gdef\currentrowstyle{#1}%
  #1\ignorespaces
}
\usepackage{threeparttable}%table with notes
  %table/figure move left
\usepackage{changepage}
  %table cmidrule
\usepackage{booktabs}

%include algorithm/flow charts
\usepackage[noend]{algorithmic}
\usepackage{algorithm}
% \usepackage{algpseudocode}
\newcommand{\LINEFORALL}[2]{%
    \STATE\algorithmicforall\ {#1}\ \algorithmicdo\ {#2} %
}
\renewcommand{\algorithmiccomment}[1]{\bgroup\hfill//~#1\egroup}

% colors for reviews
% color conflict with xthesis fixed by this: https://tex.stackexchange.com/questions/119486/another-issue-with-universitys-class-file-and-tikz-package-relates-to-queens
\usepackage{color}
\usepackage[usenames,dvipsnames]{xcolor}
\newcommand{\ion}[1]{{\bf {\textcolor{WildStrawberry}{IF:  #1}}}}
\newcommand{\luis}[1]{{\bf {\textcolor{BurntOrange}{HZ:  #1}}}}
% plotting
\usepackage{tikz}
\usetikzlibrary{matrix}
% roman letters
\newcommand{\rom}[1]{\expandafter{\romannumeral #1}}
\newcommand{\RNum}[1]{\uppercase\expandafter{\romannumeral #1\relax}}
% interesting symbols
\usepackage{pifont}
% differential d
\newcommand{\de}{\text{\rm d}}
% add appendix
\usepackage[toc,page]{appendix}
% use natbib
\usepackage{natbib}
% hyperlink reference
\RequirePackage[citecolor=blue,urlcolor=blue]{hyperref}
% THEOREMS ----------------------------------------------------------------
\usepackage{amsthm,amsmath,amsfonts,newlfont}
\theoremstyle{plain}
\newtheorem{thm}{Theorem}[section]
\newtheorem{cor}[thm]{Corollary}
\newtheorem{lem}[thm]{Lemma}
\newtheorem{prop}[thm]{Proposition}
\newtheorem{prob}[thm]{Problem}
\newtheorem{ques}[thm]{Question}
\newtheorem{ass}[thm]{Assumption}
\def\proof{{\bf{Proof.}}}
%
\theoremstyle{definition}
\newtheorem{defn}{Definition}[section]
%
\theoremstyle{remark}
\newtheorem{rem}{Remark}[section]
\newtheorem{example}{Example}[section]
%
\numberwithin{equation}{section}
\renewcommand{\theequation}{\thesection.\arabic{equation}}
%%% -----------------------------------------------------------------------





%              this is for the list of symbols page, if desired
\def\listofsymbols{\input{symbols} \clearpage}
\def\addsymbol #1: #2#3{$#1$\> \parbox{5in}{#2 \dotfill  \pageref{#3}}\\}
\def\newnot#1{\label{#1}}
%
\pagestyle{fancy}
% \fancyhead[LE,RO]{helv \thepage}
\fancyhead[L,R]{helv \thepage}
\setlength{\headheight}{15.2pt}     %%test this

%\fancyhead[LE,RO]{\thepage\hspace{2em}\footnotesize{\leftmark}}  % try this to move over header

\addtolength{\voffset}{-4em}   								% May2009 added this to move page number up a bit



\doublespacing

\lhead{}
\chead{}
\rhead{\thepage}
\lfoot{}
\cfoot{}
\rfoot{}

\renewcommand{\headrulewidth}{0 pt}          %this prints a line under the header
\renewcommand{\footrulewidth}{0  pt}            %this prints a line under the footer


\renewcommand{\contentsname}{\normalsize{Table of Contents}}
\renewcommand{\listfigurename}{\normalsize{List of Figures}}
\renewcommand{\listtablename}{\normalsize{List of Tables}}
\renewcommand{\bibname}{\normalsize{Bibliography}}
\renewcommand{\indexname}{\normalsize{Index}}


% bea02152011 - commented out the following line.  This line reformats the table of contents subsection appearance,
% to look like n.nn (chapterNumber dot sectionNumber subsectionNumber)instead of
% n.n.n  (chapterNumber dot sectionNumber dot subsectionNumber ) 
%   \renewcommand{\thesubsection}{\thesection\arabic{subsection}}


% you may need to change the pdfpagewidth to pagewidth
% and pdfpageheight to pageheight

% if not using pdflatex to produce output, you may need to change to pagewidth and pageheight variables.
%\pagewidth 8.5in
%\pageheight 11in 
\pdfpagewidth 8.5in
\pdfpageheight 11in 
%


\setlength{\oddsidemargin}{0.5in}  
\setlength{\evensidemargin}{0.5in} 
\setlength{\textwidth}{6.0in}
\setlength{\textheight}{8.5in} 
\setlength{\topmargin}{0.in}    
\setlength{\headheight}{0.5in}
\setlength{\headwidth}{6.0in}
\setlength{\headsep}{0.65in}                       			% change from .25in to .5 in   May2009 
\setlength{\parindent}{12mm}



%%%%%%%%%%%%%%%%%%%%%%%%%%%%%%%%%%%%%%%%%%%
%
% If you only want to create output that includes only certain chapters
%
%\includeonly{chapter2}
%%%%%%%%%%%%%%%%%%%%%%%%%%%%%%%%%%%%%%%%%%%



%%%%%%%%%%%%%%%%%%%%%%%%%%%%%%%%%%%%%%%%%%%%
%%%
%%%  This begins the frontmatter of the document, everything preceding the body 
%%%
%%%%%%%%%%%%%%%%%%%%%%%%%%%%%%%%%%%%%%%%%%%%


\begin{document}

\pagenumbering{roman}

\thesistitlepage

\thesiscopyrightpage              
	
\addcontentsline{toc}{chapter}{Abstract}          		 				% added May2009
\thesisabstract

\addcontentsline{toc}{chapter}{Dedication}  							 % added May2009


\addcontentsline{toc}{chapter}{Acknowledgments}   				% added May2009
\thesisacknowledgments

\makeatletter \renewcommand{\@dotsep}{10000} \makeatother

%\changepage{textheight}{textwidth}%									 added May2009 to raise up the TOC on page and fix margin
%  {evensidemargin}{oddsidemargin}%
%  {columnsep}{topmargin}%
%  {headheight}{headsep}%
%  {footskip}


\renewcommand{\contentsname}{\normalsize{Table of Contents}}      % added this line May2010 to fix issue with 
\tableofcontents                                                 %toc appearing in too large a font size when used in Linux
                     
										


\newpage		% added June 2009

\addcontentsline{toc}{chapter}{List of Tables}     					% added May2009
\listoftables

\newpage     % added June2009
\addcontentsline{toc}{chapter}{List of Figures}						% added May2009
\listoffigures

%%%
%    This produces the list of symbols.  the dots have been left in to show how it would look with dots.
%    The formatting of the table of contents, list of symbols and list of figures is up to the author, 
%    but they should all match and be similar in format.    The list of symbols is optional, it can be un-commented
%    out if desired.
%
%   \newpage
%   \chapter*\normalsize\textbf{List of Symbols\hfill}   %/hfill
%   \addcontentsline{toc}{chapter}{List of Symbols}
%   \listofsymbols


%%%%%%%%%%%%%%%%%%%%%%%%%%%%%%%%%%%%%%%%%%%%
%%%
%%%  This begins the body of the document
%%%
%%%%%%%%%%%%%%%%%%%%%%%%%%%%%%%%%%%%%%%%%%%%


\newpage
\pagenumbering{arabic}

% chapters can be included in separate files, or in this main program.  To create a chapter in a separate file,
% put the text in a .tex file, including the \chapter title.  See the examples for chapter1 and chapter2 below
\chapter{Introduction}

The volatility, as a measurement of the risk, is one of the most important element in the study of finance. \cite{whaley_derivatives_1993} proposes that by allowing investors to hedge directly against shifts in volatility, these securities enable investors to avoid the costs of dynamically adjusting positions for changes in volatility and serve to make the market more complete. During the past few years, derivatives written on volatility have developed rapidly, in which options are clearly the fundamental type of derivatives. However, unlike options on stocks, pricing volatility options can be challenging by its mean-revering property. On the basis of \cite{cox_theory_1985}, \cite{hull_pricing_1987}, \cite{heston_closed-form_1993}, \cite{grunbichler_valuing_1996} proposes Mean-Reverting Square Root(MRSR) model; \cite{chan_empirical_1992} then generalizes Mean-Reverting constant elasticity of variance(MRCEV) model and \cite{gatheral_consistent_2008} proposes the Double CEV model to capture the dynamics of volatility. \cite{grunbichler_valuing_1996} gives a closed-form solution for options under the MRSR model, but there do not exist solutions to the following two models. The aim of our paper is to use the solution of square-root mean-reverting model to approximate the other two \footnote{Strictly speaking, for double CEV model, we mainly consider Heston plus CEV model, that is $\gamma_1=\frac{1}{2}$. Details are discussed in \ref{sec: 3.2}}{models}.

\cite{heath_variance_2002}(HP) first introduce a diffusion operator integral (DOI) method, he uses an auxiliary model and then apply Ito calculus on it to find unbiased variance reduction estimators for the true model, he also adapt DOI method to be employed in conjunction with PDE methods and test his method with Heston model. Similar to the idea of HP, \cite{kristensen_adding_2011}(KM) doesn't use auxiliary model to serve as a variance reduction estimator, but apply Ito-Taylor expansions on the difference of auxiliary model and true model to  create increasingly improved refinements. KM apply his method on several fields including bond pricing under CIR model and option pricing under stochastic volatility models.

Based on HP and KM's work, we extend their methods to price options under mean-revering models. This paper is organized as following, in chapter \ref{ch2}, we illustrate DOI method and KM's expansion method; In chapter \ref{ch3}, we discuss our method on approximating option prices under mean-reverting CEV model and {Heston plus CEV model}\footnote{This is a special case of double CEV model with $\gamma_1=\frac{1}{2}$ in the first CEV model.}, including the auxiliary model selection, techniques of taking partial derivatives; In chapter \ref{ch4}, we show our numerical results and we put our conclusion in chapter \ref{ch5}.

\chapter{Method Description}

In this section, the origin DOI method, JDOI method and approximation method based on DOI method are described. In section \ref{sec: 2.1},, we introduce the origin DOI method. In section \ref{sec: 2.2}, we illustrate the approximation method proposed by \cite{kristensen_adding_2011}. In section \ref{sec: 2.3}, we discuss our estimator to price American options based on JDOI method.

\section{The DOI Variance Reduction Method}
\label{sec: 2.1}

Consider a multi-factor model, in which a $d$-dimensional vector of state variables $X(t)$ on a filtered probability space$(\Omega,\mathcal F, \mathbb Q)$ satisfies the following Stochastic Differential Equations(SDEs)

\begin{equation}\label{general model}
    dX(t) = \mu(t, X(t)) dt + \sigma(t, X(t)) dW(t)
\end{equation}

\noindent where $\mu(t,X(t))$ and $\sigma(t, X(t))$ are drift and diffusion functions under the risk-neutral measure $\mathbb Q$, which also satisfies appropriate growth and Lipschiz conditions such that equation(\ref{general model}) admits a unique strong solution and is Markovian; $W(t)$ is a $d$-dimensional standard Brownian Motion and $t \in [0,T]$.

% Besides, we also need to consider the stopping time formulation such that we can apply DOI method to path-dependent options, we shall discuss it later in section \ref{sec: 2.3} when pricing American options.

% Let $G(t, x)$ be the payoff function of derivatives written on $X(t)$ and current state $X(t) = x$, $t\in[0,T]$, assume $G(t,x)$ satisfies that 

% \begin{equation}
%     \mathbb{E}_{t, x}\left[\sup _{0 \leq u \leq T-t}\left|e^{-\int_t^{t+u} R(s,X(s)) ds} G(T, X(t+u))\right|\right]<\infty
% \end{equation}

% \begin{equation}
%     V(t,x) = \mathbb{E}^{t,x}[e^{-\int_t^T R(s,X(s)) ds} G(T, X(T))]
% \end{equation}

Let $V(t,x)$ be the value function of European option written on $X(T)$ with current state $X(t)=x$, $G(t, x)$ be the payoff function. We define the infinitesimal generator $\mathcal{L}$ associated with equation(\ref{general model})to be

\begin{equation}\label{general inf gen}
    \begin{aligned}
        (\mathcal{L} V)(t, x)&=\frac{\partial V}{\partial t} + \sum_{i=1}^{d} \mu_i(t, x) \frac{\partial V}{\partial x}+\frac{1}{2} \sum_{i=1}^{d}\sum_{j=1}^{d} (\sigma(t,x) \sigma^{\intercal}(t,x))_{i,j} \frac{\partial^2 V}{\partial x_i x_k}
    \end{aligned}
\end{equation}

Let $R(t,x)$ be the instantaneous short-term interest rate, $Q(t,x)$ the instantaneous coupon rate, combining with equation(\ref{general model}) and equation(\ref{general inf gen}), the price of European option $V$ is a solution to the following partial differential equation(PDE)

\begin{equation}
    LV(x,t) = (R(x,t) - Q(x,t))V(x,t)
\end{equation}
\noindent with boundary condition $V(T,x(T)) = G(T,X(T))$. It's easily seen that under risk neutral measure $\mathbb Q$, the instantaneous option price change is equal to the price gain in saving account minus paid coupon. 

Next we consider to use a $m$-dimensional($m \leq d$) process $\bar{X}(t)$ which is a simpler process to approximate the price of option. $\bar{X}(t)$ satisfies the following SDE

\begin{equation}\label{approx model}
    \begin{aligned}
        &d\bar{X}(t)= \begin{cases}   \bar{\mu}_i(t, \bar{X}(t)) dt + \bar{\sigma}_i(t, \bar{X}(t)) dW(t) & 1 \leq i \leq m \\
        0 & \text { otherwise }\end{cases}
        \end{aligned}
\end{equation}

\noindent where $\bar{\mu}(t, \bar{X}(t))$ and $\bar{\sigma}(t, \bar{X}(t))$ are drift and diffusion functions, and they are also assumed to satisfy appropriate conditions such that equation(\ref{approx model}) admits a unique strong solution and is Markovian.

With this new process, the European option written on $\bar{X}(t)$ is given by

\begin{equation}
    \bar{V} = \mathbb{E}_{t, x}\left[e^{-\int_t^{t+u} R(s,\bar{X}(s)) ds} G(T, \bar{X}(t+u))\right]
\end{equation}

\noindent and assume the following additional integrability condition

\begin{equation}
    \mathbb{E}_{t, x}\left[\sup _{0 \leq u \leq T-t}\left|e^{-\int_t^{t+u} R(s,\bar{X}(s)) ds} G(T, \bar{X}(t+u))\right|\right]<\infty
\end{equation}

Additionally, strong Markovian arguments imply that the European-style option $\bar{V}$ satisfies

\begin{equation}
    L\bar{V}(x,t) + q(t,x) = R(x,t)\bar{V}(x,t)
\end{equation}

It's easily seen that

\begin{equation}
    V(t,x) = \bar{V} + \mathbb{E}_{t,x}\left[\right]
\end{equation}

\section{Approximation Method based on DOI method}
\label{sec: 2.2}

\section{JDOI method}
\label{sec: 2.3}
  
\normalsize 

% \begin{figure}[htp]
% \centering
% \includegraphics{sin_x.jpg}
% \caption{Transverse momentum distributions}\label{fig:erptsqfit}
% \end{figure}



\chapter{Approximations of VIX options}

% In this section, we use the methods illustrated before to price options under several volatility models. In section \ref{sec: 3.1}, we approximate option prices with 1-dimensional volatility processes; In section \ref{sec: 3.2}, method is given to price options under double CEV model. In section \ref{sec: 3.3}, while pricing path-dependent options like American options, we use the DOI estimator and shows that it's able reduce simulations' variance.

\section{Approximating options under mean-reverting CEV model}
\label{sec: 3.1}

\subsection{Drawbacks of using Black-Scholes model as an auxiliary model}
\cite{chan_empirical_1992} proposes the mean-reversion CEV model, in which volatility follows

$$
    d V_{t}=\left(\alpha+\beta V_{t}\right) d t+\sigma V_{t}^{\gamma} d W_{t}
$$

\noindent when $\beta$ is negative, this model has mean-reverting property. We can rewrite it to be

\begin{equation}\label{mr}
    d V_t=\kappa(m - V_t) d t+\sigma V^{\gamma}_t d W_t
\end{equation}

\noindent where $\kappa$ is the speed of mean-reversion, $m$ is the long-run mean. A natural idea is to use Black-Scholes model as auxiliary model as mentioned in \cite{kristensen_adding_2011}, then apply their method to approximate the VIX option price under mean-reverting CEV model. Denote $\mathcal{L}$ and $\mathcal{L}^{\text{BS}}$ to be infinitesimal generators of mean-reverting CEV model and Black-Scholes model respectively

$$
\begin{aligned}
    \mathcal{L} w&= \frac{\partial w}{\partial t}+\kappa(m - V) \frac{\partial w}{\partial V}+\frac{1}{2} \sigma^{2} V^{2\gamma} \frac{\partial^{2} w}{\partial V^{2}} \\
    \mathcal{L}^{\text{BS}} w &= \frac{\partial w}{\partial t}+rV \frac{\partial w}{\partial V}+\frac{1}{2} \sigma^{2} V^2 \frac{\partial^{2} w}{\partial V^{2}}
\end{aligned}
$$

\noindent The mis-pricing term for using Black-Scholes model is then

$$\delta^{\text{BS}} = (\mathcal{L} - \mathcal{L}^{\text{BS}}) w^{\text{BS}} = (\kappa - r)V \frac{\partial w^{\text{BS}}}{\partial t} + \kappa m \frac{\partial w^{\text{BS}}}{\partial t} + \sigma^{2} (V - V^{2 \gamma}) \frac{\partial^{2} w^{\text{BS}}}{\partial V^{2}} $$

\noindent with the solution of Black-Scholes model $w^{\text{BS}}$. Note that $\delta^{\text{BS}}$ contains theta and gamma of option. Their differences in Black-Scholes model and mean-reverting model determines that we have to use other auxiliary models.

Take call option prices under $\gamma=\frac{1}{2}$ in model\eqref{mr} as an example. This model is known as mean square root mean-reverting model proposed by \cite{grunbichler_valuing_1996}.


\begin{figure}[ht]
    \centering
    \includegraphics[width=10cm]{./figures/call2T.png}
    \caption{Call option price with regard to time to maturity}\label{call2t}
\end{figure}

\begin{figure}[ht]
    \centering
    \includegraphics[width=10cm]{./figures/call2V.png}
    \caption{Call option price with regard to volatility}\label{call2v}
\end{figure}

From figure \ref{call2t}, we can find that in contrast to Black-Scholes model, the value of call option price under mean-reverting model is not always increasing as time to maturity increases; From figure \ref{call2v}, by contrast, the call option price does not converge to zero as volatility goes to zero. In addition, \cite{grunbichler_valuing_1996} also shows that $V$ has less influence of the current value of the call option than in Black-Scholes model. For these reasons, we conclude that Black-Scholes model is not an appropriate auxiliary model and in the next section, we discuss that using the square root mean-reverting model as the auxiliary model.

\subsection{Using square root mean-reverting model as auxiliary model}

Recall the mean-reverting CEV model with $\gamma=\frac{1}{2}$

\begin{equation}
    d V_t=\kappa(m - V_t) d t+\sigma \sqrt{V_t} d W_t
\end{equation}

We are going to use it as our auxiliary model as it captures the mean-reverting property of general mean-reverting CEV models. \cite{grunbichler_valuing_1996} gives an explicit solution to this model. Denote the call option price $\bar{w}$ with strike $K$, constant risk-free rate $r$, time to maturity $T$ and no expected premium for volatility risk is paid, its price is given by

\begin{equation}\label{aux call price}
    \begin{aligned}
        \bar{w}=&  e^{ -(\kappa+r) T} V Q(x K ; \nu+4, \lambda) \\
        &+ m e^{-r T}(1-e^{-\kappa T}) Q(xK ; \nu+2, \lambda) \\
        &-e^{-r T} K Q(x K; \nu, \lambda)
        \end{aligned}
\end{equation}

\noindent where

$$
\begin{aligned}\label{para}
    &x=\frac{4 \kappa}{\sigma^{2}(1-e^{-\kappa T})} \\
    &\nu=\frac{4 \kappa m}{\sigma^{2}}, \\
    &\lambda= e^{-\kappa T}x V
    \end{aligned}
$$

\noindent and $Q(xK ; \nu+i, \lambda)$ is the complementary distribution function for the non-central chi-squared density with $\nu + i$ degrees of freedom and non-centrality parameter $\lambda$.

Define the infinitesimal generators $\bar{\mathcal{L}}$ for square root mean-reverting model and $\mathcal{L}$ for mean-reverting CEV model

\begin{equation}\label{inf gen1}
    \begin{aligned}
        \mathcal{L} w&= \frac{\partial w}{\partial t}+\kappa(m - V) \frac{\partial w}{\partial V}+\frac{1}{2} \sigma^{2} V^{2\gamma} \frac{\partial^{2} w}{\partial V^{2}} \\
        \bar{\mathcal{L}} w &= \frac{\partial w}{\partial t}+\kappa(m - V) \frac{\partial w}{\partial V}+\frac{1}{2} \sigma^{2} V \frac{\partial^{2} w}{\partial V^{2}}
    \end{aligned}
\end{equation}

Subtract infinitesimal generators in equation\eqref{inf gen1}, we get the mis-pricing formula for using square root mean-reverting model

$$
\delta = (\mathcal{L} - \bar{\mathcal{L}}) \bar{w} = \frac{1}{2} \sigma^{2} (V^{2\gamma} - V) \frac{\partial^{2} w}{\partial V^{2}}
$$

\noindent We can then use the approximation formula discussed in \ref{sec: 2.2} to price call options under mean-reverting CEV model\footnote{Put options can be priced easily in the same way}

\begin{equation} \label{cev approx formula}
    w_{N}(t, x)=\bar{w}(t,x)+\sum_{n=0}^{N} \frac{(T-t)^{n+1}}{(n+1) !} \delta_{n}(t, x)
\end{equation}

\noindent where

\begin{equation}\label{mispricing}
    \begin{aligned}
        &\delta_0 = \delta = \frac{1}{2} \sigma^{2} (V^{2\gamma} - V) \frac{\partial^{2} w}{\partial V^{2}} \\
        &\delta_{n}(t, x)=L \delta_{n-1}(t, x)- r\delta_{n-1}(t, x)
        \end{aligned}
\end{equation}

Finally we get a closed form approximating formula for call options under mean-reverting CEV model. But notice that the call price \eqref{aux call price} contains non-square chi square distribution functions, applying infinitesimal generator $\mathcal{L}$ on it can be a hard point and in the next section we are going to talk about how to derive partial derivatives of distribution function $Q(xK; \nu+i, \lambda)$.

\subsection{Method to Calculate Derivatives In Expansions}

In this section, methods to calculate closed-form partial derivatives of call option price $\bar{w}$ to time $t$ and volatility $V$. Our method is based on the recurrence relation of non-central chi-square distribution proposed by \cite{cohen_noncentral_1988}, which is

\begin{equation}\label{pdf diff}
    \begin{gathered}
        \frac{\partial p(xK;\nu,\lambda)}{\partial (xK)}=\frac{1}{2}[-p(xK ; \nu, \lambda)+p(xK ; \nu-2, \lambda)]\\
        \frac{\partial p(xK;\nu,\lambda)}{\partial \lambda}=\frac{1}{2}[-p(xK ; \nu, \lambda)+p(xK ; \nu+2, \lambda)] \\
    \end{gathered}
\end{equation}

\noindent where $p(xK;\nu,\lambda)$ is the Probability Density Function(PDF) of non-central chi-square distribution. From the relationship between Complementary Cumulative Distribution Function(CCDF) $Q(xK;\nu,\lambda)$, Cumulative Distribution Function(CDF) $F(xK;\nu,\lambda)$, and PDF we know that

\begin{equation}\label{CCDF2x}
    \begin{aligned}
        \frac{\partial Q(xK; \nu, \lambda)}{\partial (xK)}&=\frac{\partial[1-F(xK; \nu, \lambda)]}{\partial (xK)} \\ 
        &=-\frac{\partial F(xK; \nu, \lambda)]}{\partial (xK)}\\
        &= -p(xK;\nu,\lambda)
    \end{aligned}
\end{equation}

\noindent Rewrite the second equation in \eqref{pdf diff}, we get

\begin{equation}\label{pdf trans}
    \begin{aligned}
        \frac{\partial p(xK;\nu,\lambda)}{\partial \lambda}&=\frac{1}{2}[-p(xK ; \nu, \lambda)+p(xK ; \nu+2, \lambda)] \\
        &=-\frac{1}{2}[-p(xK ; \nu+2, \lambda)+p(xK ; \nu, \lambda)]\\
        &= -\frac{\partial p(xK;\nu+2,\lambda)}{\partial (xK)}
    \end{aligned}
\end{equation}

\noindent Integrate both sides of \eqref{pdf trans} with respect to $xK$ and combine with \eqref{CCDF2x}, we can derive the partial derivative of CDF to non-central parameter $\lambda$

\begin{equation}
    \begin{aligned}
        \frac{\partial}{\partial \lambda} F(xK;\nu,\lambda)&=-\frac{\partial}{\partial (xK)}F(xK;\nu+2,\lambda) \\
        &= -p(xK;\nu+2,\lambda)
    \end{aligned}
\end{equation}

\noindent Finally we get the partial derivative of CCDF to non-central parameter $\lambda$

\begin{equation}\label{CCDF2lambda}
    \begin{aligned}
        \frac{\partial Q(xK; \nu, \lambda)}{\partial \lambda}&=\frac{\partial[1-F(xK; \nu, \lambda)]}{\partial \lambda} \\ 
        &=-\frac{\partial F(xK; \nu, \lambda)]}{\partial \lambda}\\
        &= p(xK;\nu+2,\lambda)
    \end{aligned}
\end{equation}

Until now we can summarize that the derivatives of CCDF and PDF are all combinations of PDFs with change of degrees of freedom. Without loss of accuracy, we make the degrees of freedom in PDF be consistent with call option solution in \eqref{aux call price}, that is for $p(xK;\nu+i, \lambda)$, we let $i \in [0,4]$. Use the non-central chi-square property by \cite{cohen_noncentral_1988} to do the following transformation

\begin{equation}\label{trans}
    \begin{aligned}
        p(xK ; \nu-2, \lambda)&=\frac{\lambda}{xK} p(xK ; \nu+2, \lambda)+\frac{v-2}{xK} p(xK ; \nu, \lambda) \\
        p(xK ; \nu+6, \lambda)&=\frac{xK}{\lambda} p(xK ; \nu+2, \lambda)-\frac{\nu+2}{\lambda} p(xK ; \nu+4, \lambda)
    \end{aligned}
\end{equation}

Next we use the results above to calculate delta and gamma of auxiliary call option price $\bar{w}$. Recall the parameter $xK$, $\nu$ and $\lambda$ in \eqref{para}, where

\begin{equation}
    \begin{aligned}
        &x=\frac{4 \kappa}{\sigma^{2}(1-e^{-\kappa T})} \\
        &\nu=\frac{4 \kappa m}{\sigma^{2}}, \\
        &\lambda= e^{-\kappa T}x V
    \end{aligned}
\end{equation}

\noindent Then we use chain rule calculate the following auxiliary derivatives

\begin{equation}
    \begin{aligned}
        \frac{\partial Q(xK; \nu, \lambda)}{\partial V}&= \frac{\partial Q}{\partial x}\frac{\partial x}{\partial V} + \frac{\partial Q}{\partial \lambda} \frac{\partial \lambda}{\partial V} \\
        &=0 + x e^{-\kappa T} p(x ; \nu+2, \lambda)\\
        &= x e^{-\kappa T} p(x ; \nu+2, \lambda)
    \end{aligned}
\end{equation}

\noindent Thus delta is given by

\begin{equation}
    \begin{aligned}
        \Delta_{\bar{w}}=&  e^{ -(\kappa+r) T} Q(x K ; \nu+4, \lambda) + e^{ -(\kappa+r) T}V \cdot x e^{-\kappa T} p(x K ; \nu+6, \lambda)\\
        &+ m e^{-r T}(1-e^{-\kappa T}) \cdot x e^{-\kappa T} p(x ; \nu+2, \lambda) -e^{-r T} K \cdot x e^{-\kappa T} p(x ; \nu+2, \lambda)
        \end{aligned}
\end{equation}

\noindent Using \eqref{trans} to substitute $p(xK;\nu+6,\lambda)$ and simplify the equation

\begin{equation}
    \begin{aligned}
        \Delta_{\bar{w}}=&  e^{ -(\kappa+r) T} Q(x K ; \nu+4, \lambda) \\
        &+ e^{ -(\kappa+r) T} \lambda \left[\frac{xK}{\lambda} p(xK ; \nu+2, \lambda)-\frac{\nu+2}{\lambda} p(xK ; \nu+4, \lambda)\right]\\
        &+ m e^{-r T}(1-e^{-\kappa T}) \cdot \frac{4 \kappa}{\sigma^{2}(1-e^{-\kappa T})} e^{-\kappa T} p(x ; \nu+2, \lambda) -e^{-(r+\kappa) T} K  p(x ; \nu+2, \lambda) \\
        =& e^{ -(\kappa+r) T} [Q(x K ; \nu+4, \lambda)-2p(x K ; \nu+4, \lambda)] 
        \end{aligned}
\end{equation}

% \begin{figure}[ht]
%     \centering
%     \includegraphics[width=10cm]{./figures/delta_comparison.png}
%     \caption{delta comparison, computed by our formula and discretisation}\label{delta comparison}
% \end{figure}

% \begin{figure}[ht]
%     \centering
%     \includegraphics[width=10cm]{./figures/delta_diff.png}
%     \caption{delta differences}\label{delta diff}
% \end{figure}

\begin{figure}[!tbp]
    \centering
    \subfloat[delta comparison]{\includegraphics[width=0.8\textwidth]{./figures/delta_comparison.png}\label{delta comparison}}
    \hfill
    \subfloat[delta differences between two methods]{\includegraphics[width=0.8\textwidth]{./figures/delta_diff.png}\label{delta diff}}
    \caption{Deltas are calculated by our formula, and finite difference method. The parameters used are $T=0.3$,$\alpha = 0.60$, $\beta = 4.00$, $\sigma = 0.133$, $r = 0.05$, and $K= 0.15$.  }
  \end{figure}


  \begin{figure}[!tbp]
    \centering
    \subfloat[gamma comparison]{\includegraphics[width=0.8\textwidth]{./figures/gamma_comparison.png}\label{gamma comparison}}
    \hfill
    \subfloat[gamma differences between two methods]{\includegraphics[width=0.8\textwidth]{./figures/gamma_diff.png}\label{gamma diff}}
    \caption{gammas are calculated by our formula, and finite difference method. The parameters used are $T=0.3$,$\alpha = 0.60$, $\beta = 4.00$, $\sigma = 0.133$, $r = 0.05$, and $K= 0.15$.  }
  \end{figure}

\noindent Similarly, we calculate another auxiliary derivative

\begin{equation}
    \begin{aligned}
        \frac{\partial p(xK;\nu,\lambda)}{\partial V} &= \frac{\partial p}{\partial (xK)}\frac{\partial (xK)}{\partial V} + \frac{\partial p}{\partial \lambda} \frac{\partial \lambda}{\partial V} \\
        &= \frac{x e^{-\kappa T}}{2} [-p(xK ; \nu, \lambda)+p(xK ; \nu+2, \lambda)]
    \end{aligned}
\end{equation}

\noindent As a result, gamma of $\bar{w}$ is then

\begin{equation}\label{gamma}
    \begin{aligned}
        \Gamma_{\bar{w}}&= e^{ -(\kappa+r) T} \left[x e^{-\kappa T} p(x ; \nu+6, \lambda)-2 \cdot \frac{x e^{-\kappa T}}{2} [-p(xK ; \nu, \lambda)+p(xK ; \nu+2, \lambda)]\right] \\
        &= xe^{ -(2\kappa+r) T}p(xK;nu+4,\lambda)
    \end{aligned}
\end{equation}

To apply infinitesimal generator on mis-pricing formula, we still need to calculate partial derivatives of PDF to time $t$. Define the following auxiliary functions

\begin{equation}
    \begin{aligned}
        \frac{\partial (x K)}{\partial t}&= \frac{-\kappa e^{-\kappa T}}{1 - e^{-\kappa T}} \cdot  xK\\
        \frac{\partial \lambda}{\partial t}& =\frac{-\kappa e^{-\kappa T}}{1 - e^{-\kappa T}} \cdot  xV
    \end{aligned}
\end{equation}

\noindent Then partial derivatives of PDF to $t$ is given by

\begin{equation}
    \begin{aligned}
        \frac{\partial p(xK; \nu, \lambda)}{\partial t}&= \frac{\partial p}{\partial (xK)}\frac{\partial (xK)}{\partial t} + \frac{\partial p}{\partial \lambda} \frac{\partial \lambda}{\partial t} \\
        &= \frac{-\kappa x e^{-\kappa T}}{2(1 - e^{-\kappa T})} \left[Vp(xK ; \nu+2, \lambda) - (K+V) p(xK ; \nu, \lambda) + K p(xK ; \nu-2, \lambda)\right]\\
    \end{aligned}
\end{equation}

From \eqref{mispricing} we know that the mis-pricing formula $\delta = \frac{1}{2} \sigma^2 (V^{2\gamma}-V) \Gamma_{\bar{w}}$, all terms in which have been solved from above. In essence, to apply Ito-Taylor expansions on $\delta$, we use the following algorithm as used in calculating delta and gamma:

\begin{enumerate}
    \item Combining previous auxiliary partial derivatives, use chain rule to apply infinitesimal generator on mis-pricing formula.
    \item Substitute PDFs with noncentral parameter $\nu+i$ where $\nu \notin [0,4]$.
    \item Back to step 1, apply higher order infinitesimal generators.
\end{enumerate}

Therefore, we illustrate a solution to implement approximation method on volatility options under mean-reverting CEV model. The expansions in approximating formula can be computed once for all, we can solve it manually or use symbolic language for higher orders. All terms in the result is explicit expect non-central chi-square PDFs, we plug $p(xK;\nu+i,\lambda)$ into the result at last.

\section{Approximating options under double Heston model}

\cite{gatheral_consistent_nodate} proposes volatility with double mean-reverting dynamics

$$
    \begin{aligned}
        d V_t &=-\kappa\left(V_t-V^{\prime}(t)\right) d t+\eta_{1} V^{\prime \alpha}_t  d W_1(t) \\
        d V^{\prime}_t &=-c\left(V^{\prime}_t-m\right) d t+\eta_{2} V^{\prime \beta}_t d W_{2}(t)
    \end{aligned}
$$

\noindent where $\alpha, \beta \in [\frac{1}{2},1]$.

\begin{itemize}
    \item It's called Double Heston model in the case $\alpha=\beta=\frac{1}{2}$.
    \item The case $\alpha=\beta=1$ Double Log-normal model.
    \item And the general Double CEV model.
\end{itemize}

From our previous work, we can use the same auxiliary model to price options with $V_t$ following heston dynamics and $V^{\prime}_t$ following any mean-reverting CEV process, we call it one Heston one CEV model. This model is given by

\begin{equation}\label{heston cev}
    \begin{aligned}
        d V_t &=-\kappa\left(V_t-V^{\prime}(t)\right) d t+\eta_{1} \sqrt{V_t} d W_1(t) \\
        d V^{\prime}_t &=-c\left(V^{\prime}_t-m\right) d t+\eta_{2} V^{\prime \beta}_t d W_{2}(t)
    \end{aligned}
\end{equation}

Define infinitesimal generator $\mathcal{L}$ for \eqref{heston cev} and $\bar{\mathcal{L}}$ for square root mean-reverting model

\begin{equation}\label{inf gen2}
    \begin{aligned}
        \mathcal{L} w&= \frac{\partial w}{\partial t}+\kappa(V^{\prime} - V) \frac{\partial w}{\partial V}+\frac{1}{2} \eta_1^{2} V \frac{\partial^{2} w}{\partial V^{2}} \\
        &+ \frac{\partial w}{\partial t}+c(m^{\prime} - V^{\prime}) \frac{\partial w}{V^{\prime}}+\frac{1}{2} \eta_2^{2} V \frac{\partial^{2} w}{V^{\prime 2}}\\
        \bar{\mathcal{L}} w &= \frac{\partial w}{\partial t}+\kappa(m - V) \frac{\partial w}{\partial V}+\frac{1}{2} \eta_1 V \frac{\partial^{2} w}{\partial V^{2}}
    \end{aligned}
\end{equation}

\noindent Mis-pricing formula for it is then

$$
\begin{aligned}
    \delta = (\mathcal{L} - \bar{\mathcal{L}}) \bar{w} &= \kappa(V^{\prime}-m)\frac{\partial w}{\partial V} \\
    &= \kappa(V^{\prime}-m) \Gamma_{\bar{w}}
\end{aligned}
$$

\noindent where $\Gamma_{\bar{w}}$ is given in \eqref{gamma}.

\chapter{Numerical Results}

In this section, we will show our approximation results for mean-reverting CEV model and Heston plus CEV model, we expand this method with corrective terms up to 3. We utilize a symbolic library of \footnote{Codes used for this paper can be accessed through my \href{https://github.com/ywang408/master-thesis-code}{github}}{python} \href{https://www.sympy.org/en/index.html}{sympy} to apply expansions, and Monte Carlo simulations with 200 steps and 100000 paths as our benchmark because there's no existing pricing formula for these models. To evaluate the accuracy of our results, we use two kinds of figures, the first kind is the direct comparison between benchmarks and our approximation results, the second one is the relative differences between benchmarks and our results. Besides, we also attach detailed results in the Appendix \ref{mrcev}.

\section{Volatility option prices under mean-reverting CEV model}

For options under mean-reverting CEV model,

\begin{equation}\label{numerical1}
  \begin{cases}
    d V_t=\kappa(m - V_t) d t+\sigma_{\text{CEV}} V^{\gamma}_t d W_t &\text{true model}\\
    d V_t=\kappa(m - V_t) d t+\sigma_{\text{CEV}}V_0^{\gamma-\frac{1}{2}} \sqrt{V_t} d W_t &\text{auxiliary model}
  \end{cases}
\end{equation}

\noindent we use the same mean-reverting parameters as \cite{grunbichler_valuing_1996} used in his model, the parameters are $\kappa=4$, $\theta=2$. Besides, we set the nuisance parameter $\sigma_0 = \sigma_{\text{CEV}} V_0^{\gamma-\frac{1}{2}}$, where $V_0=V(t)$, which is the initial value of volatility at time t, see equation \eqref{numerical1}. We test our approximation method with different constant elasticity parameters, the main idea of setting these parameters is that for small $\gamma$, which enlarges the importance of $V$ in the CEV part, we use a small $\sigma$; Whereas for large $\gamma$ we set a large $\sigma$. 

For figure \ref{mrcev res1} and figure, our parameters are $\sigma=0.15$, $\gamma=0.3$, and with different maturities $T=0.3$, $T=0.5$. We can find that in \ref{price comparison1} and \ref{price comparison2} our results with corrective terms $N=3$ are very accurate; Figure \ref{price diff1} and figure \ref{price diff2} show the relative error with different corrective terms. We can find that the results with highest corrective terms outperform other results, which implies that keep applying Ito-Taylor expansions on the mis-pricing formula can create increasingly improved refinements and provide us with more and more accurate results.

\begin{figure}[ht]
    \centering
    \subfloat[price comparison]{\includegraphics[width=0.5\textwidth]{./figures/T=0.3,K=0.15, kappa=4,m=0.2, sigma=0.15, gamma=0.3.csv price.png}\label{price comparison1}}
    \hfill
    \subfloat[relative error]{\includegraphics[width=0.5\textwidth]{./figures/T=0.3,K=0.15, kappa=4,m=0.2, sigma=0.15, gamma=0.3.csv error.png}\label{price diff1}}
    \caption{Parameters are $T=0.3,K=0.15, \kappa=4,\theta=0.2, \sigma=0.15, \gamma=0.3$}
  \end{figure}\label{mrcev res1}

\begin{figure}[ht]
  \centering
  \subfloat[price comparison]{\includegraphics[width=0.5\textwidth]{./figures/T=0.5,K=0.15, kappa=4,m=0.2, sigma=0.15, gamma=0.3.csv price.png}\label{price comparison2}}
  \hfill
  \subfloat[relative error]{\includegraphics[width=0.5\textwidth]{./figures/T=0.5,K=0.15, kappa=4,m=0.2, sigma=0.15, gamma=0.3.csv error.png}\label{price diff2}}
  \caption{Parameters are $T=0.5,K=0.15, \kappa=4,\theta=0.2, \sigma=0.15, \gamma=0.3$}
\end{figure}\label{mrcev res2}
\centering

Parameters for figure \ref{price comparison3} and figure \ref{price comparison4},our parameters are $\sigma=0.6$, $\gamma=0.75$, and with different maturities $T=0.3$, $T=0.5$. Similarly, our method still provide accurate results.

\begin{figure}[ht]
  \centering
  \subfloat[price comparison]{\includegraphics[width=0.5\textwidth]{./figures/T=0.3,K=0.15, kappa=4,m=0.2, sigma=0.6, gamma=0.75.csv price.png}\label{price comparison3}}
  \hfill
  \subfloat[relative error]{\includegraphics[width=0.5\textwidth]{./figures/T=0.3,K=0.15, kappa=4,m=0.2, sigma=0.6, gamma=0.75.csv error.png}\label{price diff3}}
  \caption{Parameters are $T=0.3,K=0.15, \kappa=4,\theta=0.2, \sigma=0.6, \gamma=0.75$}
\end{figure}

\begin{figure}[ht]
    \centering
    \subfloat[price comparison]{\includegraphics[width=0.5\textwidth]{./figures/T=0.5,K=0.15, kappa=4,m=0.2, sigma=0.6, gamma=0.75.csv price.png}\label{price comparison4}}
    \hfill
    \subfloat[relative error]{\includegraphics[width=0.5\textwidth]{./figures/T=0.5,K=0.15, kappa=4,m=0.2, sigma=0.6, gamma=0.75.csv error.png}\label{price diff4}}
    \caption{Parameters are $T=0.5,K=0.15, \kappa=4,\theta=0.2, \sigma=0.6, \gamma=0.75$}
\end{figure}


One may observe that when KM use Black-Scholes model as auxiliary model to price options under Heston model, for deep in-the-money options, relative error always converge to 0 no matter how many corrective terms are applied. That is because in his case delta of option is very close to 1 and vega is close to 0, which means option prices are mainly driven by underlying stocks' prices, and volatility has no influence on option prices. Besides, using Black-Scholes model makes mis-pricing term depend on gamma, while gamma of deep in-the-money options is also close to 0, meaning that their mis-pricing terms don't affect option prices at all. As a result, their figures show that all results' relative errors are converging to 0 as stock price increases.

However, in our model, underlying assets follow mean-reverting CEV model. \cite{grunbichler_valuing_1996} mention that when volatility $V$ is above its long-term mean, mean-reversion property implies the expected future value of V will be lower than its current value, making the expected payoff for a volatility call can be less than its current intrinsic value. The property of options under Black-Scholes world doesn't hold here, recall that before we set $\sigma_0 = \sigma_{\text{CEV}}v_0^{\gamma-\frac{1}{2}}$. Obviously when $V_0$ is large, $\sigma_{\text{CEV}} V^{\gamma}_t < \sigma_{\text{CEV}}V_0^{\gamma-\frac{1}{2}} \sqrt{V}_t$, causing the loss of accuracy in our auxiliary model. It gives an explanation why the relative error of our method is slightly larger than 0 for deep in-the-money options. Additionally, using our method to price deep out-of-money options can also be challenging. The loss of accuracy for approximating non-central chi-square distribution functions would be magnified when option price is very small.


\section{Volatility option prices under Heston puls CEV model}

For Heston plus CEV model, our parameters are $r=0.05$, $K=0.15$,$\kappa_1=4$, $\kappa_2=2$, $\theta_2=0.2$, $\sigma_1=0.3$, $\sigma_2=0.8$, $\gamma=1.6$, $\rho=0.5$ and different maturities $T-t=0.3, T-t=0.5$. We set the nuisance parameters $\theta_0 = \theta$, where $V_2(t)=0.2$ is the spot value for volatility of volatility at time $t$.

\begin{figure}[ht]
  \centering
  \subfloat[price comparison]{\includegraphics[width=0.5\textwidth]{./figures/heston cev T=0.3,K=0.15, kappa=4,kappa2 =2,theta=0.2, sigma1=0.3,sigam2=0.8, gamma=1.6.csv price.png}\label{2d price comparison1}}
  \hfill
  \subfloat[relative error]{\includegraphics[width=0.5\textwidth]{./figures/heston cev T=0.3,K=0.15, kappa=4,kappa2 =2,theta=0.2, sigma1=0.3,sigam2=0.8, gamma=1.6.csv error.png}\label{2d price diff1}}
  \caption{Heston plus CEV model result 1}
\end{figure}

\begin{figure}[ht]
  \centering
  \subfloat[price comparison]{\includegraphics[width=0.5\textwidth]{./figures/heston cev T=0.5,K=0.15, kappa=4,kappa2 =2,theta=0.2, sigma1=0.3,sigam2=0.8, gamma=1.6.csv price.png}\label{2d price comparison2}}
  \hfill
  \subfloat[relative error]{\includegraphics[width=0.5\textwidth]{./figures/heston cev T=0.5,K=0.15, kappa=4,kappa2 =2,theta=0.2, sigma1=0.3,sigam2=0.8, gamma=1.6.csv error.png}\label{2d price diff2}}
  \caption{Heston plus CEV model result 2}
\end{figure}

As is seen in \ref{2d price comparison1}, applying our method under 2-dimensional model can still create relatively accurate results. Unlike mean-reverting CEV model, under Heston plus CEV model our relative error is now converging to 0. This is because here the initial value of volatility doesn't enter mis-pricing term $\delta = \kappa_1(V_2-\theta_2) \Delta_{\bar{w}}$.

Besides, we notice that for \ref{2d price comparison2} when $T=0.5$, our results aren't accurate compared to other situations. Though in this case results become more accurate as we take more corrective terms. We may predict that if applying higher order expansions we could get more precise results. However, this raises a limit of KM's method, number of terms grow exponentially in the final pricing formula as we apply higher order expansions, causing the running time of calculation increasing dramatically. Such condition leads to reconsidering nuisance parameters and auxiliary models.

\chapter{Conclusion}

In this paper, we introduce approximation method proposed by \cite{david_variance_nodate} and \cite{kristensen_adding_2011}. Based on their work, we extend this method to price options under mean-reverting CEV model and Heston plus CEV models. Selections of auxiliary models and corresponding mis-pricing formula are discussed, we also illustrate techniques to calculate partial derivatives of non-central chi-square distribution functions when using square root mean-reverting as auxiliary model. Finally, we discuss our numerical results and explain the constraints of our method. In all, numerical results show that our method is efficient and accurate.
% \subsection{Derivatives Valuation}

In this section, methods to calculate closed-form partial derivatives of call option price $\bar{w}$ to time $t$ and volatility $V$. Our method is based on the recurrence relation of non-central chi-square distribution proposed by \cite{cohen_noncentral_1988}, which is

\begin{equation}\label{pdf diff}
    \begin{gathered}
        \frac{\partial p(xK;\nu,\lambda)}{\partial (xK)}=\frac{1}{2}[-p(xK ; \nu, \lambda)+p(xK ; \nu-2, \lambda)]\\
        \frac{\partial p(xK;\nu,\lambda)}{\partial \lambda}=\frac{1}{2}[-p(xK ; \nu, \lambda)+p(xK ; \nu+2, \lambda)] \\
    \end{gathered}
\end{equation}

\noindent where $p(xK;\nu,\lambda)$ is the Probability Density Function(PDF) of non-central chi-square distribution. From the relationship between Complementary Cumulative Distribution Function(CCDF) $Q(xK;\nu,\lambda)$, Cumulative Distribution Function(CDF) $F(xK;\nu,\lambda)$, and PDF we know that

\begin{equation}\label{CCDF2x}
    \begin{aligned}
        \frac{\partial Q(xK; \nu, \lambda)}{\partial (xK)}&=\frac{\partial[1-F(xK; \nu, \lambda)]}{\partial (xK)} \\ 
        &=-\frac{\partial F(xK; \nu, \lambda)]}{\partial (xK)}\\
        &= -p(xK;\nu,\lambda)
    \end{aligned}
\end{equation}

\noindent Rewrite the second equation in \eqref{pdf diff}, we get

\begin{equation}\label{pdf trans}
    \begin{aligned}
        \frac{\partial p(xK;\nu,\lambda)}{\partial \lambda}&=\frac{1}{2}[-p(xK ; \nu, \lambda)+p(xK ; \nu+2, \lambda)] \\
        &=-\frac{1}{2}[-p(xK ; \nu+2, \lambda)+p(xK ; \nu, \lambda)]\\
        &= -\frac{\partial p(xK;\nu+2,\lambda)}{\partial (xK)}
    \end{aligned}
\end{equation}

\noindent Integrate both sides of \eqref{pdf trans} with respect to $xK$ and combine with \eqref{CCDF2x}, we can derive the partial derivative of CDF to non-central parameter $\lambda$

\begin{equation}\label{CCDF2lambda}
    \begin{aligned}
        \frac{\partial}{\partial \lambda} F(xK;\nu,\lambda)&=-\frac{\partial}{\partial (xK)}F(xK;\nu+2,\lambda) \\
        &= -p(xK;\nu+2,\lambda)
    \end{aligned}
\end{equation}

\noindent Finally we get the partial derivative of CCDF to non-central parameter $\lambda$

\begin{equation}\label{CCDF2lambda}
    \begin{aligned}
        \frac{\partial Q(xK; \nu, \lambda)}{\partial \lambda}&=\frac{\partial[1-F(xK; \nu, \lambda)]}{\partial \lambda} \\ 
        &=-\frac{\partial F(xK; \nu, \lambda)]}{\partial \lambda}\\
        &= p(xK;\nu+2,\lambda)
    \end{aligned}
\end{equation}

Until now we can summarize that the derivatives of CCDF and PDF are all combinations of PDFs with change of degrees of freedom. Without loss of accuracy, we make the degrees of freedom in PDF be consistent with call option solution in \eqref{aux call price}, that is for $p(xK;\nu+i, \lambda)$, we let $i \in [0,4]$. Use the non-central chi-square property by \cite{cohen_noncentral_1988} to do the following transformation

\begin{equation}\label{trans}
    \begin{aligned}
        p(xK ; \nu-2, \lambda)&=\frac{\lambda}{xK} p(xK ; \nu+2, \lambda)+\frac{v-2}{xK} p(xK ; \nu, \lambda) \\
        p(xK ; \nu+6, \lambda)&=\frac{xK}{\lambda} p(xK ; \nu+2, \lambda)-\frac{\nu+2}{\lambda} p(xK ; \nu+4, \lambda)
    \end{aligned}
\end{equation}

Next we use the results above to calculate delta and gamma of auxiliary call option price $\bar{w}$. Recall the parameter $xK$, $\nu$ and $\lambda$ in \eqref{params}, where

\begin{equation}
    \begin{aligned}
        &x=\frac{4 \kappa}{\sigma^{2}(1-e^{-\kappa T})} \\
        &\nu=\frac{4 \kappa m}{\sigma^{2}}, \\
        &\lambda= e^{-\kappa T}x V
    \end{aligned}
\end{equation}

\noindent Then we use chain rule calculate the following auxiliary derivatives

\begin{equation}
    \begin{aligned}
        \frac{\partial Q(xK; \nu, \lambda)}{\partial V}&= \frac{\partial Q}{\partial x}\frac{\partial x}{\partial V} + \frac{\partial Q}{\partial \lambda} \frac{\partial \lambda}{\partial V} \\
        &=0 + x e^{-\kappa T} p(x ; \nu+2, \lambda)\\
        &= x e^{-\kappa T} p(x ; \nu+2, \lambda)
    \end{aligned}
\end{equation}

\noindent Thus delta is given by

\begin{equation}
    \begin{aligned}
        \bar{w}=&  e^{ -(\kappa+r) T} Q(x K ; \nu+4, \lambda) + e^{ -(\kappa+r) T}V \cdot x e^{-\kappa T} p(x K ; \nu+6, \lambda)\\
        &+ m e^{-r T}(1-e^{-\kappa T}) \cdot x e^{-\kappa T} p(x ; \nu+2, \lambda) -e^{-r T} K \cdot x e^{-\kappa T} p(x ; \nu+2, \lambda)
        \end{aligned}
\end{equation}

\noindent Using \eqref{trans} to substitute $p(xK;\nu+6,\lambda)$ and simplify the equation

\begin{equation}
    \begin{aligned}
        \Delta_{\bar{w}}=&  e^{ -(\kappa+r) T} Q(x K ; \nu+4, \lambda) \\
        &+ e^{ -(\kappa+r) T} \lambda \left[\frac{xK}{\lambda} p(xK ; \nu+2, \lambda)-\frac{\nu+2}{\lambda} p(xK ; \nu+4, \lambda)\right]\\
        &+ m e^{-r T}(1-e^{-\kappa T}) \cdot \frac{4 \kappa}{\sigma^{2}(1-e^{-\kappa T})} e^{-\kappa T} p(x ; \nu+2, \lambda) -e^{-(r+\kappa) T} K  p(x ; \nu+2, \lambda) \\
        =& e^{ -(\kappa+r) T} [Q(x K ; \nu+4, \lambda)-2p(x K ; \nu+4, \lambda)] 
        \end{aligned}
\end{equation}

\noindent Similarly, we calculate another auxiliary derivative

\begin{equation}
    \begin{aligned}
        \frac{\partial p(xK;\nu,\lambda)}{\partial V} &= \frac{\partial p}{\partial (xK)}\frac{\partial (xK)}{\partial V} + \frac{\partial p}{\partial \lambda} \frac{\partial \lambda}{\partial V} \\
        &= \frac{x e^{-\kappa T}}{2} [-p(xK ; \nu, \lambda)+p(xK ; \nu+2, \lambda)]
    \end{aligned}
\end{equation}

\noindent As a result, gamma of $\bar{w}$ is then

\begin{equation}
    \begin{aligned}
        \Gamma_{\bar{w}}&= e^{ -(\kappa+r) T} \left[x e^{-\kappa T} p(x ; \nu+6, \lambda)-2 \cdot \frac{x e^{-\kappa T}}{2} [-p(xK ; \nu, \lambda)+p(xK ; \nu+2, \lambda)]\right] \\
        &= xe^{ -(2\kappa+r) T}p(xK;nu+4,\lambda)
    \end{aligned}
\end{equation}

To apply infinitesimal generator on mis-pricing formula, we still need to calculate partial derivatives of PDF to time $t$. Define the following auxiliary functions

\begin{equation}
    \begin{aligned}
        \frac{\partial (x K)}{\partial t}&= \frac{-\kappa e^{-\kappa T}}{1 - e^{-\kappa T}} \cdot  xK\\
        \frac{\partial \lambda}{\partial t}& =\frac{-\kappa e^{-\kappa T}}{1 - e^{-\kappa T}} \cdot  xV
    \end{aligned}
\end{equation}

\noindent Then partial derivatives of PDF to $t$ is given by

\begin{equation}
    \begin{aligned}
        \frac{\partial p(xK; \nu, \lambda)}{\partial t}&= \frac{\partial p}{\partial (xK)}\frac{\partial (xK)}{\partial t} + \frac{\partial p}{\partial \lambda} \frac{\partial \lambda}{\partial t} \\
        &= \frac{-\kappa x e^{-\kappa T}}{2(1 - e^{-\kappa T})} \left[Vp(xK ; \nu+2, \lambda) - (K+V) p(xK ; \nu, \lambda) + K p(xK ; \nu-2, \lambda)\right]\\
    \end{aligned}
\end{equation}

From \eqref{mispricing} we know that the mis-pricing formula $\delta = \frac{1}{2} \sigma^2 (V^{2\gamma}-V) \Gamma_{\bar{w}}$, all terms in which have been solved from above. In essence, to apply Ito-Taylor expansions on $\delta$, we use the following algorithm as used in calculating delta and gamma:

\begin{enumerate}
    \item Combining previous auxiliary partial derivatives, use chain rule to apply infinitesimal generator on mis-pricing formula.
    \item Substitute PDFs with noncentral parameter $\nu+i$ where $\nu \notin [0,4]$.
    \item Back to step 1, apply higher order infinitesimal generators.
\end{enumerate}

% Finally, we illustrate a solution to implement approximation method on volatility options under mean-reverting CEV model. The expansions in approximating formula can be computed once for all, we can solve it manually or use symbolic language for higher orders. All terms in the result is explicit expect non-central chi-square PDFs, we plug $p(xK;\nu+i,\lamba)$ into the result at last.
% \chapter{Case Study}
% All information about formatting a dissertation can be found on the library website at www.stevens.edu/library/services/phd.html.  Formatting information gets updated occasionally, so it is always good to check the requirements before submitting your dissertation.  

% It is not required, but is strongly suggested that you make an appointment at the library to have the format of your dissertation reviewed before the final submission.  Dissertations can be rejected if they do not adhear to the formatting requirements.   

% All information about formatting a dissertation can be found on the library website at www.stevens.edu/library/services/phd.html.  Formatting information gets updated occasionally, so it is always good to check the requirements before submitting your dissertation.  
% It is not required, but is strongly suggested that you make an appointment at the library to have the format of your dissertation reviewed before the final submission.  Dissertations can be rejected if they do not adhear to the formatting requirements.   
% All information about formatting a dissertation can be found on the library website at www.stevens.edu/library/services/phd.html.  Formatting information gets updated occasionally, so it is always good to check the requirements before submitting your dissertation.  

% \begin{table}[htp]

% \begin{center}
% \begin{tabular}{|l|c|p{3.0in}|}
% \hline
% \multicolumn{3}{|c|}{Theoretical Dissertation Timeline}\\ \hline
% Taskt&Time to Finish&Notes\\ \hline
% Problem statement&10 hours&Initially very upbeat.\\ \hline
% Research&3 days&Literature search to very previous studies.\\ \hline
% Reformulation&4 hours&Presented and accepted by advisor\\ \hline
% Research&20 days&Literature search to very previous  studies.\\ \hline
% Experiments&14 days&Do some experiments and get results.\\ \hline
% Format&1 day&Understand format guidelines for paper.\\ \hline
% Write&years&Write the paper.\\ \hline
% Revise&not too long&Proof read, etc.\\ \hline
% Format&1-3 days&Verify correct report format is used.\\ \hline
% See Library&1 hour&Meet with Doris to verify formatting.\\ \hline
% Defend&1 day&Defend your research.\\ \hline
% Revise&0 hours&It was perfect the first time.\\ \hline
% Submit&1 day&Submit final dissertation to the library.\\ \hline
% \end{tabular}
% \end{center} 

% \caption{Table of Tasks}\label{tab:erptsqfit}
% \end{table}
% It is not required, but is strongly suggested that you make an appointment at the library to have the format of your dissertation reviewed before the final submission.  Dissertations can be rejected if they do not adhear to the formatting requirements.   
% All information about formatting a dissertation can be found on the library website at www.stevens.edu/library/services/phd.html.  Formatting information gets updated occasionally, so it is always good to check the requirements before submitting your dissertation.  
% It is not required, but is strongly suggested that you make an appointment at the library to have the format of your dissertation reviewed before the final submission.  Dissertations can be rejected if they do not adhear to the formatting requirements.   
% All information about formatting a dissertation can be found on the library website at www.stevens.edu/library/services/phd.html.  Formatting information gets updated occasionally, so it is always good to check the requirements before submitting your dissertation.  

% It is not required, but is strongly suggested that you make an appointment at the library to have the format of your dissertation reviewed before the final submission.  Dissertations can be rejected if they do not adhear to the formatting requirements.   



% \section{Further Analysis}  

% \subsection*\normalsize\emph{Paper}
% The original copy of the dissertation must be produced on good quality 8 1/2$''$ x 11$''$, 20 pound acid-free white paper. The paper should not be stapled, punched, bound, colored or printed on letterhead.

% The two additional copies of the dissertation can be photocopied or produced on copier or laser paper.

% \subsection*\normalsize\emph{Ink}
% Black ink only is used to print your report.  If a graph or photograph contains color, that is fine. 

% Latex is used to help in the printing of formulas.   The following formula would be difficult to reproduce in Microsoft Word: 
%  \[
%         \frac{d}{dx}\left( \int_{0}^{x} f(u)\,du\right)=f(x).
%      \]

% This Report was produced by LaTeX\footnote {\LaTeX\ is a fun typesetting program. Single space is used in footnotes.} written by Barbara Arnett.  Barbara is not an expert with LaTeX, and anyone using LaTeX does not need to use this template.  It is simply provided to help with formatting, specifically with page numbers and images.  


% %this subsection heading below will not show up in the table of contents.
% \subsection*\normalsize\emph{Graphics} 

% Graph paper may be used for original drawings, charts or illustrations. Original drawings may be in color or black and white, however, color is allowed for illustrations only; all text must appear in black. The original thesis must contain the original graphic or illustration, not a photocopy of the drawing, graphic or illustration.

% If formulas and diagrams contain subscript and superscript characters, ensure they are large enough to read when printed in the final paper. 

% \begin{figure}[htp]
% \centering
% \includegraphics{3d-cone.pdf}
% \caption{3D Cone}\label{fig:3d-cone}
% \end{figure}

% Photographs
% Glossy prints of good reproducible quality, either black or white or color may be used. Photographs can be printed on 8 1/2$''$ x 11$''$ glossy finish paper, however, margin and page number requirements as stated above still apply for pages containing photographs. When attaching photographs to paper, double-sided tape may be used which causes the least amount of damage to the original paper. 

% Graphics that are oriented in a landscape position must be done in a manner that retains the page numbering in the upper right hand corner, as shown in figure 3.2.  This can be difficult in Microsoft Word, but it is possible.  To do this in Microsoft Word, create a new section, change the page to landscape, and place the page number in a text box.  The text box then is rotated on the landscape page.    Then begin a new section and resume portrait orientation.

% \begin{sidewaysfigure}
% \centering
% \includegraphics[height=5.0in, width=7.5in]{3d-cone.pdf}
% \caption{This is an example of a landscape image within a page.  Note that the page number remains in the upper right hand corner of the page when the page is in the portrait position.}
% \end{sidewaysfigure}




% \chapter{Recommendations}

% Glossy prints of good reproducible quality, either black or white or color may be used. Photographs can be printed on 8$''$ x 11$''$ glossy finish paper, however, margin and page number requirements as stated above still apply for pages containing photographs. When attaching photographs to paper, double-sided tape may be used which causes the least amount of damage to the original paper.

% The format of footnotes and the bibliography are to be prepared in accordance with standard practice in the field in which one is working. Documentation formatting style must be discussed with and accepted by the advisor. See further information on Reference Styles below.

% The suggested typeface for dissertations is 10 to 12 point Arial. Other suggested fonts are Times New Roman and Helvetica. Script and italic fonts should not be used except as needed in the body of the paper. Consistency should be maintained throughout the paper, using the same font for all text, diagrams, etc. on all pages.

% Electronic material, such as CDs and multimedia can accompany the dissertation. Each copy should be accompanied with a labeled CD. When bound, these are placed in an envelope in the back of the dissertation.  Oversized materials (pages greater than 8 1/2$''$ x 11$''$) can be included in the paper if necessary. Oversized pages should be folded so that 1$''$ on the left hand binding edge remains, allowing the page to be opened once the paper is bound.

% The reference style used for formatting the paper must be confirmed with your department and advisor. Frequently used styles are the APA (American Psychological Association), MLA (Modern Language Association) and AMA (American Medical Association) styles. Style guides and manuals are available for each style. Page headers are not to be used.


% \section{The Focus Group}    

% Glossy prints of good reproducible quality, either black or white or color may be used. Photographs can be printed on 8 1/2$''$ x 11$''$ glossy finish paper, however, margin and page number requirements as stated above still apply for pages containing photographs. When attaching photographs to paper, double-sided tape may be used which causes the least amount of damage to the original paper.

% The format of footnotes and the bibliography are to be prepared in accordance with standard practice in the field in which one is working. Documentation formatting style must be discussed with and accepted by the advisor. See further information on Reference Styles below.
% \subsection{focusing}

% The suggested typeface for dissertations is 10 to 12 point Arial. Other suggested fonts are Times New Roman and Helvetica. Script and italic fonts should not be used except as needed in the body of the paper. Consistency should be maintained throughout the paper, using the same font for all text, diagrams, etc. on all pages.

% \begin{table}[htp]

% \begin{center}
% \begin{tabular}{|l|c|p{3.0in}|}
% \hline
% \multicolumn{2}{|c|}{Another Timeline}\\ \hline
% Taskt&Notes\\ \hline
% Problem statement&Initially very upbeat.\\ \hline
% Research&Literature search to very previous studies.\\ \hline
% Reformulation&Presented and accepted by advisor\\ \hline
% Research&Literature search to very previous  studies.\\ \hline
% Experiments&Do some experiments and get results.\\ \hline
% Format&Understand format guidelines for paper.\\ \hline
% Write&Write the paper.\\ \hline
% Revise&Proof read, etc.\\ \hline
% Format&Verify correct report format is used.\\ \hline
% See Library&Meet with Doris to verify formatting.\\ \hline
% Defend&Defend your research.\\ \hline
% Revise&It was perfect the first time.\\ \hline
% Submit&Submit final dissertation to the library.\\ \hline
% \end{tabular}
% \end{center} 
% \caption{Timeline 2}\label{tab:erptsqfit2}
% \end{table}

% Electronic material, such as CDs and multimedia can accompany the dissertation. Each copy should be accompanied with a labeled CD. When bound, these are placed in an envelope in the back of the dissertation.  Oversized materials (pages greater than 8 1/2$''$ x 11$''$) can be included in the paper if necessary. Oversized pages should be folded so that 1½" on the left hand binding edge remains, allowing the page to be opened once the paper is bound.

% The reference style used for formatting the paper must be confirmed with your department and advisor. Frequently used styles are the APA (American Psychological Association), MLA (Modern Language Association) and AMA (American Medical Association) styles. Style guides and manuals are available for each style. Page headers are not to be used.

% \chapter{Summary and Conclusion}

% Glossy prints of good reproducible quality, either black or white or color may be used. Photographs can be printed on 8 1/2$''$ x 11$''$ glossy finish paper, however, margin and page number requirements as stated above still apply for pages containing photographs. When attaching photographs to paper, double-sided tape may be used which causes the least amount of damage to the original paper.

% The format of footnotes and the bibliography are to be prepared in accordance with standard practice in the field in which one is working. Documentation formatting style must be discussed with and accepted by the advisor. See further information on Reference Styles below.

% The suggested typeface \cite{henderson_intellectual_2009} for dissertations is 10 to 12 point Arial. Other suggested fonts are Times New Roman and Helvetica. Script and italic fonts should not be used except as needed in the body of the paper. Consistency should be maintained throughout the paper, using the same font for all text, diagrams, etc. on all pages.

% Electronic material, such as CDs and multimedia can accompany the dissertation. Each copy should be accompanied with a labeled CD. When bound, these are placed in an envelope in the back of the dissertation.  Oversized materials (pages greater than 8 1/2$''$ x 11$''$) can be included in the paper if necessary. Oversized pages should be folded so that 1½" on the left hand binding edge remains, allowing the page to be opened once the paper is bound.

% The reference style used for formatting the paper must be confirmed with your department and advisor. Frequently used styles are the APA (American Psychological Association), MLA (Modern Language Association) and AMA (American Medical Association) styles. Style guides and manuals are available for each style. Page headers are not to be used.

\begin{appendix}
\chapter{Appendix A Mean-Reverting CEV model Numerical Results}
\label{mrcev}
\noindent Appendices at the end of a dissertation are optional, and depend on the content of the dissertation. There can be one or more appendicies, however they should retain the page numbering requirements for dissertations.  Any concerns about the formatting of an appendix should be brought to Doris Oliver, who can direct you how to format your appendix if you have questions.

\begin{center}
\begin{tabular}{|l|c|p{3.0in}|}
\hline
\multicolumn{3}{|c|}{Theoretical Dissertation Timeline}\\ \hline
Taskt&Time to Finish&Notes\\ \hline
Problem statement&10 hours&Initially very upbeat.\\ \hline
Research&3 days&Literature search to very previous studies.\\ \hline
Reformulation&4 hours&Presented and accepted by advisor\\ \hline
Research&20 days&Literature search to very previous  studies.\\ \hline
Experiments&14 days&Do some experiments and get results.\\ \hline
Format&1 day&Understand format guidelines for paper.\\ \hline
Write&years&Write the paper.\\ \hline
Revise&not too long&Proof read, etc.\\ \hline
Format&1-3 days&Verify correct report format is used.\\ \hline
See Library&1 hour&Meet with Doris to verify formatting.\\ \hline
Defend&1 day&Defend your research.\\ \hline
Revise&0 hours&It was perfect the first time.\\ \hline
Submit&1 day&Submit final dissertation to the library.\\ \hline
\end{tabular}
\end{center} 

\begin{table}
    \centering
    \caption{T=0.3,K=0.15, kappa=4,m=0.2, sigma=0.15, gamma=0.3.csv}
    \begin{tabular}{rrrrrr}
    \toprule
      vol &       mc &       w1 &       w2 &       w3 &       w4 \\
    \midrule
    0.100 & 0.023570 & 0.024113 & 0.023960 & 0.024476 & 0.023380 \\
    0.125 & 0.029589 & 0.029760 & 0.029686 & 0.029921 & 0.029421 \\
    0.150 & 0.036036 & 0.036013 & 0.035989 & 0.036045 & 0.035969 \\
    0.175 & 0.042805 & 0.042709 & 0.042710 & 0.042686 & 0.042771 \\
    0.200 & 0.049793 & 0.049705 & 0.049716 & 0.049678 & 0.049750 \\
    0.225 & 0.056940 & 0.056892 & 0.056904 & 0.056881 & 0.056891 \\
    0.250 & 0.064213 & 0.064192 & 0.064201 & 0.064195 & 0.064168 \\
    0.275 & 0.071525 & 0.071552 & 0.071558 & 0.071564 & 0.071534 \\
    0.300 & 0.078864 & 0.078944 & 0.078948 & 0.078957 & 0.078941 \\
    0.325 & 0.086237 & 0.086351 & 0.086353 & 0.086361 & 0.086360 \\
    0.350 & 0.093613 & 0.093765 & 0.093765 & 0.093771 & 0.093778 \\
    0.375 & 0.101000 & 0.101181 & 0.101181 & 0.101184 & 0.101192 \\
    0.400 & 0.108382 & 0.108598 & 0.108598 & 0.108600 & 0.108606 \\
    \bottomrule
    \end{tabular}
    \end{table}
\end{appendix}



% The bibliography is below.  The formatting for the bibliography is not provided here.  Formatting for references can be in 
% the writing style set by the advisor.  Common styles are APA, MLA and Chicago style. 
%
%    if you prefer not to use BibTex, include a line with   \chapter{References} for your bibliography and do not use
%  \bibliography{myrefs} below
%
% 
%
% added this to use bibTex for the bibliography.  If doing bibliography manually, delete

\bibliographystyle{chicago}
\setlength{\bibsep}{0.0pt} % single space in bib, March2018

\newpage
\addcontentsline{toc}{chapter}{Bibliography}		% added May2009
% \bibliography{myrefs}

% For single spacing in bib items, March 2018
\begingroup\singlespacing
\bibliography{ref.bib}
\endgroup





\end{document}
 