\subsection{Derivatives Valuation}

In this section, methods to calculate closed-form partial derivatives of call option price $\bar{w}$ to time $t$ and volatility $V$. Our method is based on the recurrence relation of non-central chi-square distribution proposed by \cite{cohen_noncentral_1988}, which is

\begin{equation}\label{pdf diff}
    \begin{gathered}
        \frac{\partial p(xK;\nu,\lambda)}{\partial (xK)}=\frac{1}{2}[-p(xK ; \nu, \lambda)+p(xK ; \nu-2, \lambda)]\\
        \frac{\partial p(xK;\nu,\lambda)}{\partial \lambda}=\frac{1}{2}[-p(xK ; \nu, \lambda)+p(xK ; \nu+2, \lambda)] \\
    \end{gathered}
\end{equation}

\noindent where $p(xK;\nu,\lambda)$ is the Probability Density Function(PDF) of non-central chi-square distribution. From the relationship between Complementary Cumulative Distribution Function(CCDF) $Q(xK;\nu,\lambda)$, Cumulative Distribution Function(CDF) $F(xK;\nu,\lambda)$, and PDF we know that

\begin{equation}\label{CCDF2x}
    \begin{aligned}
        \frac{\partial Q(xK; \nu, \lambda)}{\partial (xK)}&=\frac{\partial[1-F(xK; \nu, \lambda)]}{\partial (xK)} \\ 
        &=-\frac{\partial F(xK; \nu, \lambda)]}{\partial (xK)}\\
        &= -p(xK;\nu,\lambda)
    \end{aligned}
\end{equation}

\noindent Rewrite the second equation in \eqref{pdf diff}, we get

\begin{equation}\label{pdf trans}
    \begin{aligned}
        \frac{\partial p(xK;\nu,\lambda)}{\partial \lambda}&=\frac{1}{2}[-p(xK ; \nu, \lambda)+p(xK ; \nu+2, \lambda)] \\
        &=-\frac{1}{2}[-p(xK ; \nu+2, \lambda)+p(xK ; \nu, \lambda)]\\
        &= -\frac{\partial p(xK;\nu+2,\lambda)}{\partial (xK)}
    \end{aligned}
\end{equation}

\noindent Integrate both sides of \eqref{pdf trans} with respect to $xK$ and combine with \eqref{CCDF2x}, we can derive the partial derivative of CDF to non-central parameter $\lambda$

\begin{equation}\label{CCDF2lambda}
    \begin{aligned}
        \frac{\partial}{\partial \lambda} F(xK;\nu,\lambda)&=-\frac{\partial}{\partial (xK)}F(xK;\nu+2,\lambda) \\
        &= -p(xK;\nu+2,\lambda)
    \end{aligned}
\end{equation}

\noindent Finally we get the partial derivative of CCDF to non-central parameter $\lambda$

\begin{equation}\label{CCDF2lambda}
    \begin{aligned}
        \frac{\partial Q(xK; \nu, \lambda)}{\partial \lambda}&=\frac{\partial[1-F(xK; \nu, \lambda)]}{\partial \lambda} \\ 
        &=-\frac{\partial F(xK; \nu, \lambda)]}{\partial \lambda}\\
        &= p(xK;\nu+2,\lambda)
    \end{aligned}
\end{equation}

Until now we can summarize that the derivatives of CCDF and PDF are all combinations of PDFs with change of degrees of freedom. Without loss of accuracy, we make the degrees of freedom in PDF be consistent with call option solution in \eqref{aux call price}, that is for $p(xK;\nu+i, \lambda)$, we let $i \in [0,4]$. Use the non-central chi-square property by \cite{cohen_noncentral_1988} to do the following transformation

\begin{equation}\label{trans}
    \begin{aligned}
        p(xK ; \nu-2, \lambda)&=\frac{\lambda}{xK} p(xK ; \nu+2, \lambda)+\frac{v-2}{xK} p(xK ; \nu, \lambda) \\
        p(xK ; \nu+6, \lambda)&=\frac{xK}{\lambda} p(xK ; \nu+2, \lambda)-\frac{\nu+2}{\lambda} p(xK ; \nu+4, \lambda)
    \end{aligned}
\end{equation}

Next we use the results above to calculate delta and gamma of auxiliary call option price $\bar{w}$. Recall the parameter $xK$, $\nu$ and $\lambda$ in \eqref{params}, where

\begin{equation}
    \begin{aligned}
        &x=\frac{4 \kappa}{\sigma^{2}(1-e^{-\kappa T})} \\
        &\nu=\frac{4 \kappa m}{\sigma^{2}}, \\
        &\lambda= e^{-\kappa T}x V
    \end{aligned}
\end{equation}

\noindent Then we use chain rule calculate the following auxiliary derivatives

\begin{equation}
    \begin{aligned}
        \frac{\partial Q(xK; \nu, \lambda)}{\partial V}&= \frac{\partial Q}{\partial x}\frac{\partial x}{\partial V} + \frac{\partial Q}{\partial \lambda} \frac{\partial \lambda}{\partial V} \\
        &=0 + x e^{-\kappa T} p(x ; \nu+2, \lambda)\\
        &= x e^{-\kappa T} p(x ; \nu+2, \lambda)
    \end{aligned}
\end{equation}

\noindent Thus delta is given by

\begin{equation}
    \begin{aligned}
        \bar{w}=&  e^{ -(\kappa+r) T} Q(x K ; \nu+4, \lambda) + e^{ -(\kappa+r) T}V \cdot x e^{-\kappa T} p(x K ; \nu+6, \lambda)\\
        &+ m e^{-r T}(1-e^{-\kappa T}) \cdot x e^{-\kappa T} p(x ; \nu+2, \lambda) -e^{-r T} K \cdot x e^{-\kappa T} p(x ; \nu+2, \lambda)
        \end{aligned}
\end{equation}

\noindent Using \eqref{trans} to substitute $p(xK;\nu+6,\lambda)$ and simplify the equation

\begin{equation}
    \begin{aligned}
        \Delta_{\bar{w}}=&  e^{ -(\kappa+r) T} Q(x K ; \nu+4, \lambda) \\
        &+ e^{ -(\kappa+r) T} \lambda \left[\frac{xK}{\lambda} p(xK ; \nu+2, \lambda)-\frac{\nu+2}{\lambda} p(xK ; \nu+4, \lambda)\right]\\
        &+ m e^{-r T}(1-e^{-\kappa T}) \cdot \frac{4 \kappa}{\sigma^{2}(1-e^{-\kappa T})} e^{-\kappa T} p(x ; \nu+2, \lambda) -e^{-(r+\kappa) T} K  p(x ; \nu+2, \lambda) \\
        =& e^{ -(\kappa+r) T} [Q(x K ; \nu+4, \lambda)-2p(x K ; \nu+4, \lambda)] 
        \end{aligned}
\end{equation}

\noindent Similarly, we calculate another auxiliary derivative

\begin{equation}
    \begin{aligned}
        \frac{\partial p(xK;\nu,\lambda)}{\partial V} &= \frac{\partial p}{\partial (xK)}\frac{\partial (xK)}{\partial V} + \frac{\partial p}{\partial \lambda} \frac{\partial \lambda}{\partial V} \\
        &= \frac{x e^{-\kappa T}}{2} [-p(xK ; \nu, \lambda)+p(xK ; \nu+2, \lambda)]
    \end{aligned}
\end{equation}

\noindent As a result, gamma of $\bar{w}$ is then

\begin{equation}
    \begin{aligned}
        \Gamma_{\bar{w}}&= e^{ -(\kappa+r) T} \left[x e^{-\kappa T} p(x ; \nu+6, \lambda)-2 \cdot \frac{x e^{-\kappa T}}{2} [-p(xK ; \nu, \lambda)+p(xK ; \nu+2, \lambda)]\right] \\
        &= xe^{ -(2\kappa+r) T}p(xK;nu+4,\lambda)
    \end{aligned}
\end{equation}

To apply infinitesimal generator on mis-pricing formula, we still need to calculate partial derivatives of PDF to time $t$. Define the following auxiliary functions

\begin{equation}
    \begin{aligned}
        \frac{\partial (x K)}{\partial t}&= \frac{-\kappa e^{-\kappa T}}{1 - e^{-\kappa T}} \cdot  xK\\
        \frac{\partial \lambda}{\partial t}& =\frac{-\kappa e^{-\kappa T}}{1 - e^{-\kappa T}} \cdot  xV
    \end{aligned}
\end{equation}

\noindent Then partial derivatives of PDF to $t$ is given by

\begin{equation}
    \begin{aligned}
        \frac{\partial p(xK; \nu, \lambda)}{\partial t}&= \frac{\partial p}{\partial (xK)}\frac{\partial (xK)}{\partial t} + \frac{\partial p}{\partial \lambda} \frac{\partial \lambda}{\partial t} \\
        &= \frac{-\kappa x e^{-\kappa T}}{2(1 - e^{-\kappa T})} \left[Vp(xK ; \nu+2, \lambda) - (K+V) p(xK ; \nu, \lambda) + K p(xK ; \nu-2, \lambda)\right]\\
    \end{aligned}
\end{equation}

From \eqref{mispricing} we know that the mis-pricing formula $\delta = \frac{1}{2} \sigma^2 (V^{2\gamma}-V) \Gamma_{\bar{w}}$, all terms in which have been solved from above. In essence, to apply Ito-Taylor expansions on $\delta$, we use the following algorithm as used in calculating delta and gamma:

\begin{enumerate}
    \item Combining previous auxiliary partial derivatives, use chain rule to apply infinitesimal generator on mis-pricing formula.
    \item Substitute PDFs with noncentral parameter $\nu+i$ where $\nu \notin [0,4]$.
    \item Back to step 1, apply higher order infinitesimal generators.
\end{enumerate}

% Finally, we illustrate a solution to implement approximation method on volatility options under mean-reverting CEV model. The expansions in approximating formula can be computed once for all, we can solve it manually or use symbolic language for higher orders. All terms in the result is explicit expect non-central chi-square PDFs, we plug $p(xK;\nu+i,\lamba)$ into the result at last.